\documentclass{article}

\usepackage{fontspec}

\usepackage[french]{babel}

% le paquet pour des dessins : tikz
\usepackage{tikz}

% le paquet qui permet de créer des graphiques +facilt
\usepackage{pgfplots}

% charger le document csv à utiliser
\pgfplotstableread[col sep=semicolon]{biblio/data.csv}\datatable

\begin{document}
	
	% un dessin = un evt tikzpicture
	\begin{tikzpicture}
		% draw = dessiner une forme vide (que l'outline); (0,0) = les coordonnées de départ; circle = la forme à dessiner ; 1 = la taille de la figure
		% ATTENTION : ne pas oubler les ';' à la fin d'un elt
		\draw (0,0) circle (1);
		
	\end{tikzpicture}
	
	\begin{tikzpicture}
		% filldraw = dessiner une forme vide ; fill = couleur de remplissage ; draw = couleur de trait ; ici, on fait un lien entre les coordonnées de 3 points grâce à '--' : (coord1) -- (coord2) -- (coord3)
		\filldraw[fill=purple,draw=blue] (1,0) -- (0,0) -- (0,1) -- cycle;
		
	\end{tikzpicture}
	
	\begin{figure}[!h]
		\begin{tikzpicture}
			% node : un élément entier ; entre '[]', le type de forme et les dimensions ; '(A)' le nom de l'élément; {Image} : le contenu textuel de l'élément
			\node[draw,rectangle,rounded corners=3pt,text centered,text width=2cm] (A) at (0,0) {Image};
			
			\node[draw,rectangle,rounded corners=3pt,text centered, text width=4cm] (B) at (4,0) {Texte};
			
			% avec draw, on fait le lien entre les éléments A et B
			\draw[->,thick] (A) -- (B);
		\end{tikzpicture}
	\end{figure}
	
	% faire un schéma de lemmes
	\begin{tikzpicture}
		% là c'est un peu la même sauf que il n'y a pas de '[]' vu que il n'y a pas de forme, juste du contenu textuel
		\node (O) at (0,0) {O};
		\node (O') at (0,-1) {O'};
		\node (Ab) at (-2.5,-1.8) {Ab};
		
		\draw[thick] (O) -- (O');
		\draw[thick] (O') -- (Ab);
	\end{tikzpicture}

	% faire une grille avec une courbe, boucler
	\begin{tikzpicture}
		% \draw plot permet de dessiner une courbe ; on fait un {} contenant des coordonnées, comme avec \addplot coordinates plus bas
		\draw plot[mark=*] coordinates {(1,5) (2,6) (3,7) (4,4) (5,5)};
		\draw (1,4) grid (5,7);
	
		% avec ça, on boucle sur les abscisses pour mettre des chiffres sur l'axe des abscisses
		% foreach permet de boucler sur un élément (ici \x, les abscisses)
		% in {1,2,...,5} : permet d'indiquer sur quoi et comment on boucle: {1,...,5} = l'axe des abscisses ; '1,2 définit le pas (boucler tous les 2-1 = 1 elts)
		% syntaxe abstraite : \foreach \eltName in {a,b,...,n} avec a le premier elt de l'itérable sur lequel on boucle, b le 2e, n le dernier
		% draw (\x,3.8) définit quoi écrire je crois
		% node[below]{\x} : on écrit sous l'axe des abscisses
		\foreach \x in {1,2,...,5} \draw (\x,3.8) node[below] {\x}; 
		
		% là c'est en gros la même, mais pour écire dans l'axe des ordonnées ; du coup on fait {4,5,...,7} remplace x par y, on écrit node[left] vu que les chiffres s'écrivent à gauche de l'axe
		\foreach \y in {4,5,...,7} \draw (0.8,\y) node[left] {\y};
	\end{tikzpicture}

	% faire une jolie grille avec pgfplots
	\begin{tikzpicture}
		% on importe le pkg pgfplots et on charge l'environnement axis de ce paquet ; le reste est assez transparent
		% [title=leTitreQuOnVeut]
		% grid = la forme : une grille
		% xlabel = {titreEnAbscisses}
		% ylabel = {titreEnOrdonnées}
		\begin{axis}[title={Évolution des degrés en fonction des jours},
			grid,
			xlabel={Jour},
			ylabel={Degrés}
			]
		
			% on ajoute les coordonnées de sa courbe {(coordAbs1, coordOrd1), (coordAbs2, coordOrd2), (coordAbsN, coordOrdN)};
			\addplot coordinates {(1,5) (2,6) (3,7) (4,4) (5,5)};
		\end{axis}
	\end{tikzpicture}
	
	% travailler avec des données issues d'un CSV
	\begin{tikzpicture}
		\begin{axis}[
			xtick=data,
			xticklabels from table={\datatable}{langue},
			xticklabel style= {font=\tiny},
			xlabel=Langue,
			ylabel = Nombre de lexèmes conservés
			]
			
			% bon jsp ce qu'il se passe
			\addplot table [col sep=semicolon,x expr=\coordindex,y = conservé ancien]{biblio/data.csv};
			\addplot table [col sep=semicolon,x expr=\coordindex,y = conservé moderne]{biblio/data.csv};
			
		\end{axis}
		
	\end{tikzpicture}
	
\end{document}